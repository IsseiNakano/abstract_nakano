\documentclass[oneside, 10pt, twocolumn]{jarticle}

\title{\bf{\rm}
多目的最短経路問題における
\\動的計画法に基づいた
拡張ベルマンフォード法の提案}

\author{宋研究室
\hspace{15pt}
中野 壱帥 (15715051)}
\date{}

\usepackage{amsfonts}
\usepackage{setspace}
\setstretch{1.18} % ページ全体の行間を設定
\pagestyle{empty}
\oddsidemargin -5mm
\textwidth 170mm
\topmargin -28mm
\textheight 270mm
\columnsep 5mm

% sectionの大きさ
\makeatletter
\def\section{\@startsection {section}{1}{\z@}{-3.5ex plus -1ex minus
-.2ex}{2.3 ex plus .2ex}{\large\bf}}
% sectionの行間
\renewcommand{\section}{
\@startsection{section}{1}{\z@}
{.1\Cvs \@plus.0\Cdp \@minus.1\Cdp}%  上の空き
{.1\Cvs \@plus.1\Cdp \@minus.0\Cdp}%  下の空き
{\reset@font\large\bfseries}}      %  字の大きさ
\makeatother

\begin{document}
\maketitle
\thispagestyle{empty}
%%%%%%%%%%%%%%%%%%%%%%%%%%%%%%%%%%%%%%%%%%%%%%%%%%%%%%%%%%%%%%%%%%%%%%%%%%%%%%
\section{研究背景}
現代には道路ネットワークや通信ネットワークなど様々なネットワークが存在する.
これらのネットワークには無数の組み合わせが存在するため\textbf{最適解}を求めたい.
しかし,これらのネットワークに対する最適化を行う場合,\textbf{複数の目的関数}を考慮する場合がある.
例えば,道路ネットワークでは目的地までの時間とコストを最小化する必要がある.
このように複数の目的関数値を最大化(最小化)する問題は\textbf{多目的最適化問題}と呼ばれている.
多目的最適化問題の中でも,最短経路を求める問題は\textbf{多目的最短経路問題}と呼ばれている.

多目的最適化問題においては,それぞれの目的関数が\textbf{トレードオフの関係}にある場合が存在し,
全ての目的関数値が最大(最小)となる最適解が存在するとは限らない.
そこで,\textbf{パレート最適解}の集合を求める.
一般にパレート最適解は膨大な数存在するので効率的に列挙することが必要になる.
また,調査した限り従来の多目的最短経路問題には\textbf{負の要素}を考慮した研究がなされていない.

\section{研究目的}

多目的最短経路問題において従来研究\cite{Santos}\cite{Breugem}よりも効率的にパレート解を列挙する.
また,負の要素を考慮した解を定義し,解法を提案する.

\begin{description}
  \item[目的:]
  要素が実数である多目的最短経路問題に対する解法の提案.
\end{description}

非負数の多目的最適化問題におけるパレート解列挙の複雑さを踏まえて,解法の提案を行う.
負のサイクルが存在する場合の解を定義し,効率的な解法の提案を行う.

\section{研究成果}

\begin{description}
  \item[成果1:]
  非負数における多目的最短経路問題に対して,
  ベルマンフォード法の探索順序およびデータの格納に以下の改良を加えた解法を提案した.
\end{description}

\begin{description}
  \item[1.1]
  データの格納に対して,すでに更新されたpathが更新対象とならない格納方法を実装した.
\end{description}

\begin{description}
  \item[1.2]
  比較対象や探索対象となるpathが発見しやすい格納方法を実装した.
\end{description}

\begin{description}
  \item[1.3]
  % 全体のpathに対する更新時に更新対象となっているpathのみを更新するという
  更新対象を限定した動的計画法を用いることにより,
  経由する頂点数が少ないpathから順に探索されるようにした.
\end{description}

成果1.1,成果1.2より,更新に要するpathの探索を減らし,
成果1.3より,削除されるpathの数を減らした.
以上の改良より,同じインスタンスに対して計算時間を最大90\%削減した.

\begin{description}
  \item[成果2:]
  各目的関数間の相関係数に対する問題の難しさを明らかにした.
\end{description}

頂点数や重みの範囲が同じでも,各目的関数間の相関係数によって解の数や実装時間
が変化することを実験的に証明し,有効な解法を提案した.

\begin{description}
  \item[成果3:]
  負のサイクルが存在する場合を考慮した解法を提案した.
\end{description}

多目的最短経路問題における負のサイクルの検出を行うことで負の要素を考慮し,
扱えるインスタンスの範囲を広げた.

%%%%%%%%%%%%%%%%%%%%%%%%%%%%%%%%%%%%%%%%%%%%%%%%%%%%%%%%%%%%%%%%%%%%%%%%%%%%%%
\begin{thebibliography}{9} %参考文献{載せる参考文献の数の上限}
  \bibitem{Santos}
   José Luis E. Dos Santos and José Paixão.
   Labeling Methods for the General Case of the Multi-objective
   Shortest Path Problem – A Computational Study
   . In ISCA~\textbf{61}, pp. 489-502,
   2012.
   \bibitem{Breugem}
    Thomas Breugem and Twan Dollevoet and Wilco van den Heuvel.
    Analysis of FPTASes for the multi-objective shortest path problem.
    In COR~\textbf{78}, pp. 44-58,
    2017.

\end{thebibliography}

\end{document}
