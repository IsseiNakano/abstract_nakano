\documentclass[oneside, 10pt, twocolumn]{jarticle}

\title{\bf{\rm}
多目的最短経路問題における
\\動的計画法に基づいた
拡張ベルマンフォード法の提案}

\author{宋研究室
\hspace{15pt}
中野 壱帥 (15715051)}
\date{}

\usepackage{amsfonts}
\usepackage{setspace}
\setstretch{1.18} % ページ全体の行間を設定
\pagestyle{empty}
\oddsidemargin -5mm
\textwidth 170mm
\topmargin -28mm
\textheight 270mm
\columnsep 5mm

% sectionの大きさ
\makeatletter
\def\section{\@startsection {section}{1}{\z@}{-3.5ex plus -1ex minus
-.2ex}{2.3 ex plus .2ex}{\large\bf}}
% sectionの行間
\renewcommand{\section}{
\@startsection{section}{1}{\z@}
{.1\Cvs \@plus.0\Cdp \@minus.1\Cdp}%  上の空き
{.1\Cvs \@plus.1\Cdp \@minus.0\Cdp}%  下の空き
{\reset@font\large\bfseries}}      %  字の大きさ
\makeatother

\begin{document}
\maketitle
\thispagestyle{empty}
%%%%%%%%%%%%%%%%%%%%%%%%%%%%%%%%%%%%%%%%%%%%%%%%%%%%%%%%%%%%%%%%%%%%%%%%%%%%%%
\section{研究背景}
現代には道路ネットワークや通信ネットワークなど様々なネットワークが存在する.
これらのネットワークには無数の経路や組み合わせが存在するため\textbf{最適化}した解を求めたい.
しかし,これらのネットワークに対する最適化を行う場合,\textbf{複数の目的関数}を考慮することが必要である.
例えば,道路ネットワークでは目的地までの時間とコストを最小化する必要がある.
このように複数の目的関数値を最大化(最小化)する解を求める問題は\textbf{多目的最適化問題}と呼ばれている.
多目的最適化問題の中でも,最短経路を求める問題は\textbf{多目的最短経路問題}と呼ばれている.

多目的最適化問題においては,それぞれの目的関数が\textbf{トレードオフの関係}にある場合が存在し,
全ての目的関数値が最大(最小)となる最適解が存在するとは限らない.
そこで,最適解になり得る\textbf{パレート最適解}の集合を求める.
一般にパレート最適解は膨大な数存在するので効率的に列挙することが必要になる.
また,調査した限り従来の多目的最短経路問題には\textbf{負の要素}を考慮した研究がなされていない.

\section{研究目的}

多目的最短経路問題において従来研究\cite{Santos}\cite{Breugem}よりも効率的にパレート解を列挙する.
また,負の要素を考慮した解の定義をし,解法を提案する.

\begin{description}
  \item[目的1:]
  多目的最短経路問題に対する解法の提案.
\end{description}

多目的最適化問題におけるパレート解列挙の複雑さを踏まえて,解法の提案を行う.
また,計算機を用いて解法の実験的評価を行う.

\begin{description}
  \item[目的2:]
  負の要素を考慮した解法の提案.
\end{description}

負のサイクルが存在する場合の解を定義し,効率的な解法の提案を行う.
また,計算機を用いて解法の実験的評価を行う.

\section{研究成果}

\begin{description}
  \item[成果1:]
  非負数における多目的最短経路問題に対して,
  ベルマンフォード法の探索順序およびデータの格納に以下の改良を加えた解法を提案した.
\end{description}

\begin{description}
  \item[1.1]
  データの格納に対して,すでに更新されたpathが更新対象とならない格納方法を実装した.
\end{description}

\begin{description}
  \item[1.2]
  探索する際に,頂点に対するpathが発見しやすい格納方法を実装した.
\end{description}

\begin{description}
  \item[1.3]
  全体のpathに対する更新途中で発見されたpathが更新対象にならないようにすることで,
  経由する頂点数が少ないpathから順に探索される.
\end{description}

成果1.1,成果1.2より,更新に要するpathの探索を減らし,
成果1.3より,削除されるpathの数を減らした.
以上の改良より,同じインスタンスに対して計算時間を最大93\%削減した.

\begin{description}
  \item[成果2:]
  負のサイクルが存在する場合の解を定義し,負の要素を考慮した解法を提案した.
\end{description}

負のサイクルが存在しない目的関数のみの解を定義し,
多目的最短経路問題における負のサイクルの検出を行うことで負の要素を考慮した.

%%%%%%%%%%%%%%%%%%%%%%%%%%%%%%%%%%%%%%%%%%%%%%%%%%%%%%%%%%%%%%%%%%%%%%%%%%%%%%
\begin{thebibliography}{9} %参考文献{載せる参考文献の数の上限}
  \bibitem{Santos}
   José Luis E. Dos Santos and José Paixão.
   Labeling Methods for the General Case of the Multi-objective
   Shortest Path Problem – A Computational Study
   . In ISCA~\textbf{61}, pp. 489-502,
   2012.
   \bibitem{Breugem}
    Thomas Breugem and Twan Dollevoet and Wilco van den Heuvel.
    Analysis of FPTASes for the multi-objective shortest path problem.
    In COR~\textbf{78}, pp. 44-58,
    2017.

\end{thebibliography}

\end{document}
